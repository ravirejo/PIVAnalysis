\documentclass[11pt]{report}

\usepackage{fullpage}
%\usepackage[top=1in, bottom=1in]{geometry}
\usepackage{amsfonts}
%for math symbols, Ex: \mathbb{N}
\def\Definition{what does something mean}
\usepackage{graphicx}
\usepackage[hidelinks]{hyperref}

\begin{document}

\title{PIV Analysis}
\author{Raghuvir reddy Jonnagiri}
\date{\today}
\maketitle
\thispagestyle{empty}
\clearpage

\tableofcontents
\thispagestyle{empty}
\clearpage

Hey there \hypertarget{label1}{!}
This is my \LaTeX \  report on PIV data analysis.\\
How do we start it?

Do math formula like this $(x+1)$
if you want it seperately $$(x+1)$$

for fraction use this $\displaystyle{\frac{2}{3}}$

sample table 
\begin{tabular}{|c|c|c|c|}
\hline
$input$ & 1 & 2 & 3 \\
\hline
\end{tabular} \\

Definition:\Definition

\begin{eqnarray*}
\frac{dy}{dx}&=&x^2 \\
f(y)&=&x^3/3
\end{eqnarray*}

\begin{enumerate}
\item First
\item Second
	\begin{itemize}
	\item blah
	\item blahblah
	\end{itemize}
\item[random] BLAH
\end{enumerate}

\textit{ooh...fancy}

\begin{Large}WHY ARE YOU READING THIS IN A LOUD VOICE?!..JEEEZ....\end{Large}

\begin{flushleft}
this is a bit too leftish\footnote{or is it}
\end{flushleft}
\clearpage
\setcounter{page}{1}

\section{Aim}
\section{methodology}
	\subsection{FFT}
	\subsection{Spectogram}
	\subsection{lineplots}
\section{Conclusion}
\clearpage
\begin{figure}
	\centering
	\includegraphics[width=5in]{sample.png}
	\caption{THIS IS A SAMPLE IMAGE}
	\label{fig:sample}
\end{figure}

How does that figure \ref{fig:sample} look \cite{ref1}?

You want to go back? Click \hyperlink{label}{here}
\clearpage

\begin{thebibliography}{}

\bibitem{ref1}
JRR,
Lord of the Rings,
1800 A.D.,


\end{thebibliography}

\end{document}

